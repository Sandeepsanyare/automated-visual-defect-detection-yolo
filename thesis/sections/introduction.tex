\chapter{Introduction}
\label{chap:introduction}

Industry 4.0 has revolutionized how manufacturing and quality control operate. The primary way that humans have inspected products has been by manually inspecting the product's surface for flaws. Although the best inspectors will never completely eliminate all error due to the physical limitations of the human eye (including fatigue, cognitive bias, and the inability to visually inspect everything in every inspection) human inspectors are still the primary method used to inspect surfaces for flaws today. As manufacturing continues to move at a faster pace, it will become increasingly necessary for machines to take over the role of inspectors because the speed of manufacturing far exceeds the speed at which a human inspector can visually inspect a product.

This thesis aims to develop a fully-automated surface defect detection system utilizing the You Only Look Once version 8 (YOLOv8) architecture. AOI (Automated Optical Inspection) is a critical part of modern smart factories and can detect and classify defects quickly and accurately. However, industrial surfaces present unique challenges for computer vision. These include defect patterns that have low contrast against background, defects that vary in size and shape, and defects that occur under variable lighting conditions.

In addition to the issues described above, this research identifies six different types of defect classes that were identified and included in the research: crazing, inclusion, patches, pitted surface, rolled-in scale, and scratches. All of these are unique in their morphology; e.g., "crazing" refers to a network of fine hair-like cracks that can be difficult to identify in low resolution images and may appear to be similar to surface texture. An "inclusion" refers to foreign matter that has been embedded in the product during manufacturing. An "inclusion" may appear as a black spot and can be irregularly shaped. Any successful system must be able to detect and classify all of these defect classes in spite of possible environmental contamination.

One of the goals of this research is to achieve "Zero Defect Manufacturing" (ZDM). ZDM is a goal of manufacturers who want to produce defect-free parts. To achieve ZDM, simply detecting defects is insufficient. Rather than relying solely on detecting defects, a more comprehensive framework of continuous optimization of the detection and classification processes must be developed.

To develop this framework of continuous optimization, this thesis develops a two-phase hyperparameter optimization strategy using Optuna, a Bayesian optimization framework. Through separating the optimization process into a coarse global search and a high-resolution local refinement (fine-tuning), we demonstrated that common performance plateaus can be overcome. Furthermore, we demonstrated that combining architectural efficiency with advanced data engineering techniques (i.e., tiling strategies and robust augmentation methods such as Mosaic and Mixup) are effective means of achieving optimal performance.

Following this introduction, the remainder of this document is organized along the research cycle. Section 2 reviews the relevant literature regarding the history of object detection, from early classical algorithms to state-of-the-art deep learning architectures. Section 3 examines the theoretical foundations of YOLOv8 and describes the backbones, necks, and heads of the YOLOv8 architecture. Section 4 describes the development of the proposed system design and describes the custom tiling strategy that was developed to allow processing of large-scale high-resolution image data sets. Section 5 describes the experimental methodology that was employed to develop and test the proposed system and describes the Optuna-based optimization process. Section 6 presents the analysis of the results of testing the proposed system, and highlights the success of our "Trial 20" model. Finally, Section 7 describes some of the limitations of the proposed approach and proposes several potential areas for further research.

Upon completion of this work, we anticipate providing a scalable blueprint for implementing deep-learning based defect detection systems that will meet both the accuracy and real-time requirements of modern industrial quality control.
