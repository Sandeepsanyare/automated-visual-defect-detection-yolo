\chapter{Conclusion and Future Research}
\label{chap:conclusion}

This thesis has detailed the development and rigorous optimization of a surface defect detection system based on the YOLOv8s architecture. By implementing a modular two-phase training strategy and a custom tiling inference engine, we have achieved a model that satisfies both classification accuracy and localization precision requirements for high-speed industrial manufacturing.

\section{Summary of Contributions}
-   **Tiling Inference Strategy:** Developed a robust method for handling high-resolution industrial imagery without quality loss during resizing.
-   **Optimization Pipeline:** Demonstrated that an Optuna-driven, two-phase tuning process (Coarse-to-Fine) can significantly boost mAP performance on metallic surface datasets.
-   **Validated Performance:** Achieved a mAP@50 of 0.985 and a mAP@50-95 of 0.827, outperforming standard non-optimized baselines.

\section{Directions for Future Research}
1. \textbf{YOLOv11 Integration:} Future work should explore the recently released YOLOv11 and its enhanced transformer-based backbones to further increase sensitivity to low-contrast crazing.
2. \textbf{Semi-Supervised Learning:} Labeling industrial data is expensive. Implementing a semi-supervised or self-supervised framework (e.g., using masked autoencoders) could utilize the vast amounts of unlabelled data generated by production lines.
3. \textbf{Edge Optimization:} Quantizing the model to INT8 or FP16 for deployment on NVIDIA Jetson or similar low-power edge devices using TensorRT.

In conclusion, deep learning-based AOI systems are no longer a futuristic laboratory concept but a practical, ready-to-deploy solution for modern quality assurance.
